\documentclass[letterpaper,12pt]{article}
\usepackage[letterpaper]{geometry}
\usepackage{enumerate}
\usepackage[spanish]{babel} 
\usepackage{graphicx}
\usepackage{multirow}
\usepackage{tabularx}
\usepackage{geometry}
\geometry{
  letterpaper,
  left=30mm,
  right=30mm,
  top=30mm,
  bottom=30mm,
 }


\begin{document}

\begin{large}
	\begin{center}
		UNIVERSIDAD MAYOR DE SAN SIMON\\
		DIRECCION DE ADMINISTRATIVA FINANCIERA \\
		\textbf{INFORME DE VIAJE NACIONAL O PROVINCIAL}
	\end{center}
\end{large}
\vspace{.5cm}
\begin{center}
	\begin{tabular}{|c|l|c|c|}
	\hline
	 \textbf{Nombres grupos:} & Zurita Ubaldino & \textbf{Nro CI:} & 6498429 cbba\\
	  & Marquez Fidel Alexander & & 7991594 cbba \\
	  & Pe\~na Sahonero Erikc & & 5280523 cbba\\
	  & Nina Mamani Ernesto Fanor & & 7934953 cbba\\ \hline
	\end{tabular}
\end{center}
\vspace{.5cm}
\begin{center}
		\begin{tabular}{|c|l|c|c|}
		\hline
		 \textbf{Fecha viaje ida:} & 18/10/2014 & \textbf{Tramo:} & CBBA - TJA \\ \hline
		 \textbf{Hora de salida:} & 14:00 & \textbf{Hora de llegada} & 10:00 \\
		\hline
		\end{tabular}
\end{center}
\begin{center}	
	\begin{tabular}{|c|l|c|c|}
	\hline
	 \textbf{Fecha viaje vuelta:} & 25/10/2014 & \textbf{Tramo:} & TJA - CBBA \\ \hline
	 \textbf{Hora de salida:} & 17:00 & \textbf{Hora de llegada} & 8:00 \\
	\hline
	\end{tabular}		
\end{center}

	\section{Antecedentes}
	
	La CCBOL 2014, es el congreso de Ciencias de la Computaci\'on a Nivel Bolivia en la que se presentan trabajos de investigaci\'on, se realizan conferencias y tutoriales. En esta versi\'on se ha desarrollado tambi\'en  actividades como la EXPO CCBOL, el campamento y Concurso de programaci\'on, A lo cual la Universidad Mayor de San Sim\'on ha asistido de forma activa.
	
	\section{Objetivo}
	
	Participar en la EXPO CCBOL 2014 y asistir como participante a los tutoriales y conferencias de la CCBOL 2014.
	
	\section{Desarrollo}
	
	\begin{itemize}
		\item D\'ia 19 de Octubre a horas 2:30 PM, se realizo el registro y la acreditaci\'on del equipo de investigaci\'on para el proyecto de "Rob\'otica" en las instalaciones UAJMS.
		
		\item D\'ia 20 de Octubre a horas 7:30 AM a 1:00 PM se participo de la ceremonia de apertura y la disertaci\'on del 1er Tutorial en el coliseo de la UAJMS.
		
		\item D\'ia 20 de Octubre a horas 2:30 PM a 7:15 PM se participo en las conferencias llevadas a cabo en el coliseo de la UAJMS.
		
		\item D\'ia	21 de Octubre a horas 7:30 AM a 12:15 PM se participo en el 2do Tutorial llevado a cado en el coliseo de la UAJMS.
		
		\item D\'ia 21 de Octubre a horas 2:30 PM a 7:15 PM se participo del 5to Tutorial llevado acabo en el coliseo de la UAJMS.
		
		\item D\'ia 22 de Octubre a horas 7:30 AM a 12:15 PM se participo con el proyecto de "Rob\'otica" en la EXPO CCBOL llevado a capo en inmediaciones  del Cabildo de la HAM de Tarija.
		
		\item D\'ia 22 de Octubre a horas 2:30 AM a 7:15 PM brindamos apoyo moral a los equipos "Red Pandas" y "3A's" mientras participaron del Concurso de Programaci\'on.  
		
		\item D\'ia 23 de Octubre a horas 7:30 AM a 12:15 PM se participo del 3er Tutorial llevado acabo en el coliseo de la UAJMS.
		
		\item D\'ia 23 de Octubre a horas 2:30 PM a 7:15 PM se participo de las conferencias llevadas a cabo en el coliseo de la UAJMS.
		
		\item D\'ia 24 de Octubre a horas 7:30 AM a 12:15 PM se participo del 4to Tutorial llevado acabo en el coliseo de la UAJMS.
		
		\item D\'ia 24 de Octubre a horas 2:30 PM a 7:15 PM se participo del 6to Tutorial y la ceremonia de clausura de CCBOL 2014.   

		\end{itemize}
		
		\section{Conclusiones y Resultados}

		\begin{itemize}
		
			\item Que la presentaci\'on del proyecto de "Rob\'otica" fue presentado exitosamente en la EXPO CCBOL 2014. 
			\item Se participo en los tutoriales y conferencias llevadas a cabo en la CCBOL 2014.
		\end{itemize}
		
		\section{Documentos que se adjunta}
		
		\begin{itemize}
			\item Pasajes terrestres CBBA - TJA, Flota Pilcomayo.
			\item Certificados de participaci\'on en la CCBOL 2014.
			\item Certificados de participaci\'on en la EXPO - CCBOL 2014.
			\item Pasajes de Terrestres  TJA  - CBBA, Flota Expreso del Sur.
		\end{itemize}
		\vspace{.7cm}		
		\begin{center}
			Cochabamba, 29 de octubre de 2014
		\end{center}
		\vspace{1cm}
		\begin{center}
			\begin{tabular}{cc}
			Univ. Zurita Ubaldino & Univ. Marquez Fidel Alexander \\ 
			 &  \\
			 &  \\
			Univ. Pe\~na Sahonero Erikc &  Univ. Nina Mamani Ernesto Fanor \\
		\end{tabular}				
		
		\end{center}
	
	\vspace{2cm}
	
	\begin{center}
		\textbf{Grupo Expositor: Rob\'otica}
	\end{center}
	
	cc. Direcci\'on Carrera de Ing. Inform\'atica, Direcci\'on Carrera de Ing. en Sistemas, FUL, DISU, archivo
	
\end{document}